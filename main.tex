\documentclass{book}

\usepackage{amssymb}
\usepackage{amsfonts}
\usepackage{amsmath}
\usepackage{dsfont}
\usepackage{bm}
\usepackage[a4paper, total={6in,8in}]{geometry}
\usepackage{graphicx}
\usepackage{float}
\usepackage{natbib}
\usepackage{hyperref}

\graphicspath{{figures/}}

\title{MATH30000 Project - Phonation}
\author{Will Woolfenden}

\begin{document}

\maketitle

\chapter{Introduction}

\section{Definitions}

Phonation is an extremely complex human mechanical process in which the vocal cords, being small regions of flexible tissue located in the larynx,
are subjected to complex vibrations as air is expelled from the lungs and lower airway passages towards the throat and out of the mouth.
The extremely rich and varied tones that may be produced  form the basis for the production of elongated vowel sounds.
We will not consider wider aspects of speech, such as the role of the mouth in producing consonant sounds.
The models discussed in this project are approximations to the motion of the vocal folds,
from which we aim to observe key characteristics exhibited in natural phonation.

%more

\dots

\section{Principles of non-linear dynamics and dynamical systems}

The modelling of phonation uses techniques from analysing systems of differential equations.
The study of dynamical systems, as far as we are concerned,
involves studying systems of differential equations that describe time-dependence in a model,
and provides insight into characteristics of a model such as equilibrium solutions and classes of behaviours.
In Chapter 2, we study a single mass model to approximate a vocal cord,
and in Chapter 3 we generalise into a two mass model and obtain a fourth order system of differential equations.
In all cases the equations of motion are autonomous and non-linear.
Consider the example on two variables \((x,~y)\), being
\begin{equation}
    \begin{aligned}
        \frac{\mathrm{d}x}{\mathrm{d}t} &= f(x,y) \\
        \frac{\mathrm{d}y}{\mathrm{d}t} &= g(x,y).
    \end{aligned}
\end{equation}
If the equations \(f\) and \(g\) are non-linear on their variables,
we can't do much to solve this equation.
Aside from producing numerical computations, we could look intuitively for equilibrium solutions \((x,y) = (x_0,y_0)\),
which are fixed point solutions for which all time derivatives of \(x\) and \(y\) are zero.
Hence equilibrium solutions are equivalently solutions to the homogeneous couple \(f(x,y) = 0,~g(x,y) = 0\).
Close to an equilibrium solution,
we would expect the functions' behaviour to be well approximated by their first order Taylor Series approximations, being
\begin{equation}
    \begin{aligned}
        f(x,y) &\approx f(x_0,y_0) + (x-x_0)\frac{\partial f}{\partial x}(x_0,y_0) + (y-y_0)\frac{\partial f}{\partial y}(x_0,y_0) \\
        g(x,y) &\approx g(x_0,y_0) + (x-x_0)\frac{\partial g}{\partial x}(x_0,y_0) + (y-y_0)\frac{\partial g}{\partial y}(x_0,y_0).
    \end{aligned}
\end{equation}
The constant terms disappear since \(f, g\) are zero at an equilibrium by definition. We obtain the approximation for the system local to an equilibrium solution,
\begin{equation}
    \begin{aligned}
        \frac{\mathrm{d}x}{\mathrm{d}t} &= \left(\frac{\partial f}{\partial x}(x_0,y_0)\right)(x-x_0) + \left(\frac{\partial f}{\partial y}(x_0,y_0)\right)(y-y_0) \\
        \frac{\mathrm{d}y}{\mathrm{d}t} &= \left(\frac{\partial g}{\partial x}(x_0,y_0)\right)(x-x_0) + \left(\frac{\partial g}{\partial y}(x_0,y_0)\right)(y-y_0),
    \end{aligned}
\end{equation}
which is a system of linear equations.
For convenience, we make the substitutions \(u = x-x_0,~v = y-y_0\) which are linear substitutions proportional to the variables of interest \((x,y)\).
Hence the above approximation is equivalent to the following matrix equation:
\begin{equation}
    \frac{\mathrm{d}}{\mathrm{d}t} \begin{pmatrix}
        u \\
        v
    \end{pmatrix} = \begin{bmatrix}
        \frac{\partial f}{\partial x}(x_0,y_0) & \frac{\partial f}{\partial y}(x_0,y_0) \\
        \frac{\partial g}{\partial x}(x_0,y_0) & \frac{\partial g}{\partial y}(x_0,y_0)
    \end{bmatrix} \begin{pmatrix}
        u \\
        v
    \end{pmatrix}.
\end{equation}
Letting \(\mathbf{u} = (u,v)^\mathrm{T}\) and writing \(J\) as the matrix of derivatives,
we can write the matrix equation compactly as \(\mathrm{d}\mathbf{u}/\mathrm{d}t = \mathbf{J}\mathbf{u}\).
The matrix \(J\) is the \textit{Jacobian matrix}.
The Hartman-Grobman Theorem in dynamical systems tells us a very important result,
being that the behaviour of the system near the stationary point can be determined by computing the Jacobian matrix \(\mathbf{J}\) at the equilibrium solution,
and computing the eigenvalues of the matrix.
For every equilibrium solution, we must compute the Jacobian and its eigenvalues in order to determine the behaviour of the system at all equilibrium points.
For clarity, the eigenvalues of the Jacobian are the constant terms \(\lambda\) that solve the equation \( \mathbf{J} -  \lambda \mathbf{I} = \mathbf{0}\),
where \(\mathbf{I}\) is the appropriate size identity matrix.
This also concerns the problem of finding equilibrium solutions in the first place, which is usually non-trivial.

For a more rigorous understanding of linearisation of a system of equations and an understanding of the Hartman-Grobman Theorem,
see \cite{perko_textbook_1996}, particularly sections
1.1 on simple examples of linear systems,
1.5 on two-dimensional linear systems,
1.9 on the theory of stability,
and 2.6 on linearisation.

%bit on phase portrait representations of equilibrium solutions



\bibliographystyle{unsrt}
\bibliography{sources}

\end{document}