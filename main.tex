\documentclass{book}

\usepackage{amssymb}
\usepackage{amsfonts}
\usepackage{amsmath}
\usepackage{dsfont}
\usepackage{bm}
\usepackage[a4paper, total={6in,8in}]{geometry}
\usepackage{graphicx}
\usepackage{float}
\usepackage{natbib}
\usepackage{hyperref}

\graphicspath{{figures/}}

\title{MATH30000 Project - Phonation}
\author{Will Woolfenden}

\begin{document}

\maketitle

\chapter{Introduction}

\section{The concept and process of phonation}

Phonation is a complex human mechanical process in which the vocal cords, being small regions of flexible tissue located in the larynx,
begin to vibrate as air is expelled from the lungs and lower airway passages towards the throat and out of the mouth.
As such, phonation can be regarded as the interation of two processes in the body,
being the exhalation of air from the lungs,
and the forcing of muscles in the larynx.
The extremely rich and varied tones that may be produced form the basis for the production of elongated vowel sounds.

The phonation process begins when air is expelled from the lungs.
The diaphragm contracts, applying a pressure to the lungs which causes air to be driven out.
Air travels through the airways, which have an extremely complicated, inverted tree-like structure,
collecting in the bronchi (single airway per lung) and then trachea (single airway).
Now in the upper airways, the air travels upwards from the trachea, through the larynx, and out of the body through either the nose or mouth.
It is in the larynx that the mechanical process of phonation occurs.
The glottis is the opening in the larynx between the vocal folds,
hence the glottis must be open in order for phonation to occur.
When air travels through the glottis,
a decrease in pressure may cause the vocal folds to oscillate.
When these oscillations occur,
the propagation of the vibrations lead to the production of voiced sounds.

When the glottis closes, it may be that the vocal folds come to rest,
or it could be that the glottis collapses quickly, for example in a pattern of coughing or choking.
In this case, it is important to consider how the vocal folds might collapse but then rebound open again,
and how this would occur.
The vocal folds do not deform in a linearly elastic manner \cite{alipour_2012},
meaning that the rate at which they strain is not directly proportional to the stress they are subjected to.
We could imagine that on a sudden closure of the glottis,
a forcing pressure from the lower airway drives the vocal folds apart again.
Deformations of the vocal folds on collision are one of the features considered in ``Synthesis of voiced sounds from a two-mass model of the vocal cords'' \cite{ishizaka_flanagan_1972},
which will be discussed more in the section on reviewing mathematical models.

The tones produced by the oscillations of the vocal folds produce a harmonic series,
which is the nature of all pitched sounds, such as those produced by musical instruments.
A harmonic series, in this context,
means that the waveform produced by the oscillations is a series on integer-frequency-valued periodic functions,
relative to a \textit{fundamental frequency}.
Conventionally, the mathematical definition of the harmonic series is the infinite sum:
\begin{equation}
    \hat{W} = \sum_{n=1}^{\infty}\frac{1}{n} = 1 + \frac{1}{2} + \frac{1}{3} + \mathellipsis.
\end{equation}
The overtone series is the application of the harmonic series to the frequency of oscillations.
If we have a fundamental frequency $\omega$, then we can take a series of periodic functions, in this case $\sin(x)$,
on integer multiples of the fundamental frequency. We can write this in series form as follows:
\begin{equation}
    W(t) = \sum_{n=1}^{\infty}a_n \sin(n \omega t).
\end{equation}
The constant coefficients $a_n$ define the weighting of the linear combination of the periodic functions.
A different bias on different regions of the harmonic series lead to different textures of sound.
For the signal $W(t)$, the Fourier transform of $W$ on $x$ interprets information of the signal from a time domain $t$,
and returns the data from the signal represented in the frequency domain.
While plotting $W$ against $t$ gives a representation of the signal,
the Fourier transform can give a direct plot of the strengths $a_n$ of each freqency $n\omega$. 

When formulating mathematical models,
we will make several assumptions that reduce the complexity of the model,
which make the process of constructing purely mathematical expressions much easier than they would be otherwise.
However, these approximations and assumptions also reduce the complexity of the mathematics we derive,
meaning we lose features that would arise from a model that accommodates more complex features of phonation.
There are two reasons we make these assumptions.
First of all, due to the scope of this project,
certain features must be neglected in order to finalise mathematical relationships that we can analyse in detail.
Furthermore, we want to prioritise certain aspects that we would expect to see from the model,
which in this case is the oscillatory motion of the vocal cords,
and hence it is reasonable to neglect features that are not directly important for this process.
For example, we could neglect terms in equations if their primary importance was describing how the tension on a vocal cord affects the texture of the tone it produces,
since we are not interested in terms which give extremely precise modifications to the oscillations.

\section{Structure}

The first section of this project studies the single mass model from ``Theory and measurement of snores'' \cite{gavriely_jensen_1993},
which is a model formulated to consider the factors involved in obstructive sleep apnea.
We will find that the features derived in the model are very relevant to the investigation of phonation,
and that this single mass model serves as a strong basis for more generalised models.
In the second section, we generalise the single mass model into a structure in which two masses model a single vocal cord,
where the masses have a component of stiffness coupling between them.
This model builds on the single mass model,
but also takes strong inspiration from the two mass model formulated in ``Synthesis of voiced sounds from a two-mass model of the vocal cords'' \cite{ishizaka_flanagan_1972},
which discusses the formulation of a mathematical model for the purpose of generating artificial speech using a computer.
The formulation and study of the two mass model is the largest and most important part of work contained in this project.
After having investigated the single mass model and the two mass model,
we will review the approximations and simplifications made during the formulation processes and suggest refinements to the models.
We conclude by discussing the results, which are the insights to the nature of phonation that we have gained from inspecting the behaviours of the models.

\section{Modelling phonation}

As discussed, we will make limitations to the mathematical models we consider,
to refine the model into something we can feasibly analyse while also gaining interesting results.
First of all, while inspiratory phonation (inspiration) differs from expiratory phonation (exhalation),
we will only focus on the expiratory flow case.
In the construction of the model, the cases turn out to be identical up to symmetry anyway.
We will also simplify the geometry of a model of the larynx,
reducing the problem to the flow through a cuboid, with a vocal fold modelled by a movable wall on the side of the channel.
The motion of the wall can be restricted by Hooke springs, and we can control their properties.

We will also assume either a steady or quasisteady flow.
While it is entirely possible for the flow in real speech to be turbulent and not steady,
assuming otherwise allows us to apply Bernoulli's equation for a steady flow,
which gives a simple yet powerful statement on the relationship between pressure $p$, density $\rho$ and flow velocity $\mathbf{u}$.
Restricting our investigation to incompressible fluids, namely where $\rho$ is constant,
provides similar simplifications to our analysis.

Due to the symmetry of the vocal tract,
we can simplify our analysis by only studying one side,
since accommodating both sides in our model would only provide insight under asymmetric forcing terms or initial conditions.

Further assumptions we consider are the one-dimensional motion of the modelled vocal folds,
the elasticity being linear and modelled by Hooke springs,
and {whatever else}. 

\section{Literature review for mathematical models of phonation}

The paper ``Synthesis of voiced sounds from a two-mass model of the vocal cords'' \cite[1972]{ishizaka_flanagan_1972} is arguably the most important work of research in the field of mathematical models for phonation.
In their research, Ishizaka and Flanagan devise a two-mass model for the vocal cords, and then compute results to generate synthesised voiced sounds.
The model consists of two masses to represent a vocal fold, which are stiffness coupled, comprising the wall of a channel.
A flow passes through the channel and waves propagate from the planes,
and the approximation to voiced sounds is computed as the result of the waves transmitted. %% bit more on wave physics maybe

The two mass model is an extremely simple yet insightful model,
which generates behaviours of seemingly erratic oscillations.
The main body of this project is the formulation and analysis of a two mass model.

In the mathematical study of phonation, the myoelastic and aerodynamic theories suggest the nature of the production of voiced sounds.
The myoelastic theory assumes that the vocal cords repeatedly close,
each time being driven apart by the pressure from the forced airflow from the lower airways,
and the frequency of this repeated process determines the frequency of the voiced sounds.
The aerodynamic theory instead applies the properties of pressure in a fluid flow,
imposing that a pressure drop in the glottis leads to sustained oscillations of the vocal cords.
This kind of pressure drop is often referred to as \textit{Bernoulli pressure}, which will be discussed in more depth later.
It is commonly believed that both theories are involved in the production of voiced sounds,
and the ideas are discussed by Titze in ``Comments on the myoelastic-aerodynamic theory of phonation'' \cite[1980]{titze_1980}.
Titze is an extremely prominent author in the field of voice and hearing in mathematical modelling.

The research paper ``Theory and measurement of snores'' \cite[1993]{gavriely_jensen_1993} involves a model which is not designed to approximate phonation.
Rather, its purpose is to provide mechanical insights to the factors present in obstructive sleep apnea.
However, the model itself can be applied to the investigation of phonation,
being a single mass in the wall of a channel subject to stiffness and Bernoulli pressure.
One of the authors, Oliver Jensen, is the supervisor for this project.

Mathematical models for the vocal folds are often heavily simplified, and hence some features of natural phonation are lost.
In ``Synthesis of breathy, normal, and pressed phonation using a two-mass model with a triangular glottis'' \cite[2011]{birkholz_2011},
the authors construct a model for phonation which considers a glottis of a particular shape, which is able to close gradually.
The model is able to synthesise more modes of voiced sounds than traditional models,
particularly being able to produce the range of sounds from ``breathy'' speech (soft-spoken, close to whispering) to ``pressed'' (tense, thin). 

\section{Principles of fluid mechanics}

Conventional fluid mechanics defines a flow $\mathbf{u}$ in a $2$D or $3$D space,
often as a function of position.
We will describe the $3$D case.
As well as a flow $\mathbf{u}$,
we have fluid density $\rho$ and pressure $p$,
which are often functions of position and time.
In the cases we will study,
we assume the flow is imcompressible,
meaning $\rho$ is uniform across the domain of the flow.
This may appear like a restrictive assumption,
but we still obtain rich and interesting mathematical results from the analysis of an incompressible flow.

A steady flow is a flow $u$ which is independent of time.
The streamlines in a steady flow are fixed, and are identical to the paths taken by a supposed particle placed into a flow at any time.
Bernoulli's equation for a steady flow is as follows:
\begin{equation}
    \frac{1}{2}|\mathbf{u}|^2 + \Omega + \int\frac{\mathrm{d}p}{\rho} = \mathrm{constant~along~a~streamline}.
\end{equation}
The term $\Omega$ is the potential for the body forces on the fluid, such as gravity.
For our research, we will assume there are no body forces and thus neglect $\Omega$.
Bernoulli's equation is extremely useful, since it gives us a relationship between fluid velocity and pressure in a steady flow.

Fluid mass flux $Q$ at a given cross-section is the rate at which fluid mass passes through the region,
given in terms of an integral:
\begin{equation*}
    Q = -\iint_A \rho \mathbf{u} \cdot \mathbf{n} \mathrm{d}S,
\end{equation*}
where $\mathbf{n}$ is the outer unit normal to a cross-section.
By convention, the outer unit normal points away from the volume,
hence in the opposite direction of the flow, thus we require the negative sign.
This leads to the principle of conservation of mass, where the net flux entering a volume $V$ is the same as the rate of change of fluid mass within that volume.
The net flux is the flux out minus the flow in.
This can be formally expressed as follows
\begin{equation*}
    \iiint_V \frac{\mathrm{d}\rho}{\mathrm{d}t} \mathrm{d}V = - \iint_A \rho \mathbf{u} \cdot \mathbf{n} \mathrm{d}S 
\end{equation*} 

In this research project, the models we will consider involve simplified geometry and more straightforward assumptions of mechanics in comparison to some of the literature we have acknowledged.
Fundamentally, we want to consider a fluid flow through a rigid channel, and suppose that a region of wall has freedom of motion in one dimension normal to the flow.
It is appropriate to introduce the definition of a \textit{plug flow}, being a fluid flow through a channel in which the flow is uniform over cross-sections through the channel. 

% want to finish fm section with a clear explanation of plug flow.

\section{Principles of non-linear dynamics and dynamical systems}

The modelling of phonation uses techniques from analysing systems of differential equations.
The study of dynamical systems, as far as we are concerned,
involves studying systems of differential equations that describe time-dependence in a model,
and provides insight into characteristics of a model such as equilibrium solutions and classes of behaviours.
In Chapter 2, we study a single mass model to approximate a vocal cord,
and in Chapter 3 we generalise into a two mass model and obtain a fourth order system of differential equations.
In all cases the equations of motion are autonomous and non-linear.
Consider the example on two variables \((x,~y)\), being
\begin{equation}
    \begin{aligned}
        \frac{\mathrm{d}x}{\mathrm{d}t} &= f(x,y) \\
        \frac{\mathrm{d}y}{\mathrm{d}t} &= g(x,y).
    \end{aligned}
\end{equation}
If the equations \(f\) and \(g\) are non-linear on their variables,
we can't do much to solve this equation.
Aside from producing numerical computations, we could look intuitively for equilibrium solutions \((x,y) = (x_0,y_0)\),
which are fixed point solutions for which all time derivatives of \(x\) and \(y\) are zero.
Hence equilibrium solutions are equivalently solutions to the homogeneous couple \(f(x,y) = 0,~g(x,y) = 0\).
Close to an equilibrium solution,
we would expect the functions' behaviour to be well approximated by their first order Taylor Series approximations, being
\begin{equation}
    \begin{aligned}
        f(x,y) &\approx f(x_0,y_0) + (x-x_0)\frac{\partial f}{\partial x}(x_0,y_0) + (y-y_0)\frac{\partial f}{\partial y}(x_0,y_0) \\
        g(x,y) &\approx g(x_0,y_0) + (x-x_0)\frac{\partial g}{\partial x}(x_0,y_0) + (y-y_0)\frac{\partial g}{\partial y}(x_0,y_0).
    \end{aligned}
\end{equation}
The constant terms disappear since \(f, g\) are zero at an equilibrium by definition. We obtain the approximation for the system local to an equilibrium solution,
\begin{equation}
    \begin{aligned}
        \frac{\mathrm{d}x}{\mathrm{d}t} &= \left(\frac{\partial f}{\partial x}(x_0,y_0)\right)(x-x_0) + \left(\frac{\partial f}{\partial y}(x_0,y_0)\right)(y-y_0) \\
        \frac{\mathrm{d}y}{\mathrm{d}t} &= \left(\frac{\partial g}{\partial x}(x_0,y_0)\right)(x-x_0) + \left(\frac{\partial g}{\partial y}(x_0,y_0)\right)(y-y_0),
    \end{aligned}
\end{equation}
which is a system of linear equations.
For convenience, we make the substitutions \(u = x-x_0,~v = y-y_0\) which are linear substitutions proportional to the variables of interest \((x,y)\).
Hence the above approximation is equivalent to the following matrix equation:
\begin{equation}
    \frac{\mathrm{d}}{\mathrm{d}t} \begin{pmatrix}
        u \\
        v
    \end{pmatrix} = \begin{bmatrix}
        \frac{\partial f}{\partial x}(x_0,y_0) & \frac{\partial f}{\partial y}(x_0,y_0) \\
        \frac{\partial g}{\partial x}(x_0,y_0) & \frac{\partial g}{\partial y}(x_0,y_0)
    \end{bmatrix} \begin{pmatrix}
        u \\
        v
    \end{pmatrix}.
\end{equation}
Letting \(\mathbf{u} = (u,v)^\mathrm{T}\) and writing \(J\) as the matrix of derivatives,
we can write the matrix equation compactly as \(\mathrm{d}\mathbf{u}/\mathrm{d}t = \mathbf{J}\mathbf{u}\).
The matrix \(J\) is the \textit{Jacobian matrix}.
The Hartman-Grobman Theorem in dynamical systems tells us a very important result,
being that the behaviour of the system near the stationary point can be determined by computing the Jacobian matrix \(\mathbf{J}\) at the equilibrium solution,
and computing the eigenvalues of the matrix.
The Jacobian is often a sparse matrix, and its eigenvalues are often complex.
For every equilibrium solution, we must compute the Jacobian and its eigenvalues in order to determine the behaviour of the system at all equilibrium points.
For clarity, the eigenvalues of the Jacobian are the constant terms \(\lambda\) that solve the equation \( \mathbf{J} -  \lambda \mathbf{I} = \mathbf{0}\),
where \(\mathbf{I}\) is the appropriate size identity matrix.
This also concerns the problem of finding equilibrium solutions in the first place, which is usually non-trivial.

For a more rigorous understanding of linearisation of a system of equations and an understanding of the Hartman-Grobman Theorem,
see \cite{perko_textbook_1996}, particularly sections
1.1 on simple examples of linear systems,
1.5 on two-dimensional linear systems,
1.9 on the theory of stability,
and 2.6 on linearisation.
The most important result on eigenvalues of equilibria is that an equilibrium solution is unstable if any of its eigenvalues have negative real part.

%bit on phase portrait representations of equilibrium solutions



\bibliographystyle{unsrt} % agsm is harvard
\bibliography{sources}

\end{document}