\documentclass{report}

\usepackage{amssymb}
\usepackage{amsfonts}
\usepackage{amsmath}
\usepackage{dsfont}
\usepackage{bm}
\usepackage[a4paper, total={6in,8in}]{geometry}
\usepackage{graphicx}
\usepackage{float}
\usepackage{datetime}
\usepackage{natbib}
\usepackage{hyperref}

\graphicspath{{figures/}}

\title{Models for Phonation}
\author{Will Woolfenden}

\begin{document}

\begin{titlepage}
    \begin{center}
        \vspace*{1cm}
            
        \Huge
        \textbf{Models for Phonation}
            
        \vspace{0.5cm}
        \LARGE
        Mathematical Models for Voice Production in Nonlinear Dynamics
            
        \vspace{1.5cm}
            
        \textbf{Will Woolfenden}
        
        \vspace{0.5cm}
        
        $10628968$
            
        \vfill

        Project supervisor: Prof. Oliver Jensen \\

        \vspace{0.8cm}
            %final submission
        Da submission for MATH30000 Double Project for the $2022/23$ academic year
            
        \vspace{0.8cm}
            
        \Large

        School of Mathematics\\
        The University of Manchester\\
        United Kingdom\\
        \today
    \end{center}
\end{titlepage}

\begin{abstract}
    ABSTRACT GO HERE
\end{abstract}

\chapter{Introduction}

\section{The concept and process of phonation}

Phonation is a complex human mechanical process in which the vocal cords, being small regions of flexible tissue located in the larynx,
begin to vibrate as air is expelled from the lungs and lower airway passages towards the throat and out of the mouth.
As such, phonation can be regarded as the interation of two processes in the body,
being the exhalation of air from the lungs,
and the forcing of muscles in the larynx.
The extremely rich and varied tones that may be produced form the basis for the production of elongated vowel sounds.

The phonation process begins when air is expelled from the lungs.
The diaphragm contracts, applying a pressure to the lungs which causes air to be driven out.
Air travels through the airways, which have an extremely complicated, inverted tree-like structure,
collecting in the bronchi (single airway per lung) and then trachea (single airway).
Now in the upper airways, the air travels upwards from the trachea, through the larynx, and out of the body through either the nose or mouth.
It is in the larynx that the mechanical process of phonation occurs.
The glottis is the opening in the larynx between the vocal folds,
hence the glottis must be open in order for phonation to occur.
When air travels through the glottis,
a decrease in pressure may cause the vocal folds to oscillate.
When these oscillations occur,
the propagation of the vibrations lead to the production of voiced sounds.

When the glottis closes, it may be that the vocal folds come to rest,
or it could be that the glottis collapses quickly, for example in a pattern of coughing or choking.
In this case, it is important to consider how the vocal folds might collapse but then rebound open again,
and how this would occur.
The vocal folds do not deform in a linearly elastic manner \cite{alipour_2012},
meaning that the rate at which they strain is not directly proportional to the stress they are subjected to.
We could imagine that on a sudden closure of the glottis,
a forcing pressure from the lower airway drives the vocal folds apart again.
Deformations of the vocal folds on collision are one of the features considered in ``Synthesis of voiced sounds from a two-mass model of the vocal cords'' \cite{ishizaka_flanagan_1972},
which will be discussed more in the section on reviewing mathematical models.

The tones produced by the oscillations of the vocal folds produce a harmonic series,
which is the nature of all pitched sounds, such as those produced by musical instruments.
A harmonic series, in this context,
means that the waveform produced by the oscillations is a series on integer-frequency-valued periodic functions,
relative to a \textit{fundamental frequency}.
Conventionally, the mathematical definition of the harmonic series is the infinite sum:
\begin{equation*}
    \hat{W} = \sum_{n=1}^{\infty}\frac{1}{n} = 1 + \frac{1}{2} + \frac{1}{3} + \mathellipsis.
\end{equation*}
The overtone series is the application of the harmonic series to the frequency of oscillations.
If we have a fundamental frequency $\omega$, then we can take a series of periodic functions, in this case sine,
on integer multiples of the fundamental frequency.
We can write this in series form as follows:
\begin{equation*}
    W(x,t) = \sum_{n=1}^{\infty}a_n \sin(ct) \sin(n \omega x).
\end{equation*}
For clarity, $W$ is a function on length $x$ and time $t$.
The wave speed $c$ is the speed at which the waves oscillate in time,
and $\omega$ is their fundamental frequency.
We may consider $x$ to be bounded on an interval, such as $[0, \pi]$.
In this case, we can visualise $W$ as a series of waves on a string.
The constant coefficients $a_n$ define the weighting of the linear combination of the periodic functions.
A different bias on different regions of the harmonic series lead to different textures of sound.

When formulating mathematical models,
we will make several assumptions that reduce the complexity of the model,
which make the process of constructing purely mathematical expressions much easier than they would be otherwise.
However, these approximations and assumptions also reduce the complexity of the mathematics we derive,
meaning we lose features that would arise from a model that accommodates more complex features of phonation.
There are two reasons we make these assumptions.
First of all, due to the scope of this project,
certain features must be neglected in order to finalise mathematical relationships that we can analyse in detail.
Furthermore, we want to prioritise certain aspects that we would expect to see from the model,
which in this case is the oscillatory motion of the vocal cords,
and hence it is reasonable to neglect features that are not directly important for this process.
For example, we could neglect terms in equations if their primary importance was describing how the tension on a vocal cord affects the texture of the tone it produces,
since we are not interested in terms which give extremely precise modifications to the oscillations.

\section{Structure}

The first section of this project studies the single mass model from ``Theory and measurement of snores'' \cite{gavriely_jensen_1993},
which is a model formulated to consider the factors involved in obstructive sleep apnea.
We will find that the features derived in the model are very relevant to the investigation of phonation,
and that this single mass model serves as a strong basis for more generalised models.

In the second section, we generalise the single mass model into a structure in which two masses model a single vocal cord,
where the masses have a component of stiffness coupling between them.
This model builds on the single mass model,
but also takes strong inspiration from the two mass model formulated in ``Synthesis of voiced sounds from a two-mass model of the vocal cords'' \cite{ishizaka_flanagan_1972},
which discusses the formulation of a mathematical model for the purpose of generating artificial speech using a computer.
The formulation and study of the two mass model is the largest and most important part of work contained in this project.
After having investigated the single mass model and the two mass model,
we will review the approximations and simplifications made during the formulation processes and suggest refinements to the models.
We conclude by discussing the results, which are the insights to the nature of phonation that we have gained from inspecting the behaviours of the models.

\section{Modelling phonation}

As discussed, we will make limitations to the mathematical models we consider,
to refine the model into something we can feasibly analyse while also gaining interesting results.
First of all, while inspiratory phonation (inspiration) differs from expiratory phonation (exhalation),
we will only focus on the expiratory flow case.
In the construction of the model, the cases turn out to be identical up to symmetry anyway.
We will also simplify the geometry of a model of the larynx,
reducing the problem to the flow through a cuboid, with a vocal fold modelled by a movable wall on the side of the channel.
The motion of the wall can be restricted by Hooke springs, and we can control their properties.

We will also assume either a steady or quasisteady flow.
While it is entirely possible for the flow in real speech to be turbulent and not steady,
assuming otherwise allows us to apply Bernoulli's equation for a steady flow,
which gives a simple yet powerful statement on the relationship between pressure $p$, density $\rho$ and flow velocity $\mathbf{u}$.
Restricting our investigation to incompressible fluids, namely where $\rho$ is constant,
provides similar simplifications to our analysis.

Due to the symmetry of the vocal tract,
we can simplify our analysis by only studying one side,
since accommodating both sides in our model would only provide insight under asymmetric forcing terms or initial conditions.

Further assumptions we consider are the one-dimensional motion of the modelled vocal folds,
the elasticity being linear and modelled by Hooke springs,
and a simple geometry for the vocal tract. 

\section{Motivation}

Models for phonation provide insight on the mechanics that contribute to real voice production.
Humans do not have conscious control over the vocal folds,
rather the muscles in the larynx work together with the exhalatory pressure from the lungs in order to induce glottal oscillations.
As such, producing a model for phonation means we can begin to understand the mechanics of the process.

Our goal is to formulate a model for phonation,
which we wish to can analyse to obtain mathematical descriptions of real voice production.
This can be divided into several sections.
First, we will briefly discuss the principles of mathematical modelling in fluid mechanics,
so we have a strong foundation before building our models.
When we are constructing a mathematical model,
we will be able to understand how different assumptions in the basic fluid mechanics lead to different components of the model.
Finally, this carries through into the analysis of the model,
allowing us to make links between the results a model provides,
and the mechanical properties that cause these behaviours.
If we produce a model starting with only fundamental mechanics,
then we can more easily understand which components of the model merit which features that arise in our analysis.
This means we have clear links between the physics that forms our model,
and the properties of our results.

Once we have a model,
there are many directions that can be explored to develop our study,
A mechanical model can act as a foundation for synthesis of speech \cite{ishizaka_flanagan_1972},
where we would apply mathematical principles of sound wave propagation.
Phonation is affected by plenty of factors outside of the glottal region itself,
most notably the resonance coming from the structure of the lower and upper airways as well as the mouth.
The models we explore are simplified,
and more generally study the properties of Bernoulli flow in wind tunnels.
These have applications to the study of fluid mechanics and aerodynamics of channel flows.

\section{Literature review for mathematical models of phonation}

The paper ``Synthesis of voiced sounds from a two-mass model of the vocal cords'' \cite[1972]{ishizaka_flanagan_1972} is arguably the most important work of research in the field of mathematical models for phonation.
In their research, Ishizaka and Flanagan devise a two-mass model for the vocal cords, and then compute results to generate synthesised voiced sounds.
The model consists of two masses to represent a vocal fold, which are stiffness coupled, comprising the wall of a channel.
A flow passes through the channel and sound waves propagate from the planes,
and the approximation to voiced sounds is computed as the result of the waves transmitted. %% bit more on wave physics maybe

The two mass model is an extremely simple yet insightful model,
which generates behaviours of seemingly erratic oscillations.
The main body of this project is the formulation and analysis of a two mass model.

In the mathematical study of phonation, the myoelastic and aerodynamic theories suggest the nature of the production of voiced sounds.
The myoelastic theory assumes that the vocal cords repeatedly close,
each time being driven apart by the pressure from the forced airflow from the lower airways,
and the frequency of this repeated process determines the frequency of the voiced sounds.
The aerodynamic theory instead applies the properties of pressure in a fluid flow,
imposing that a pressure drop in the glottis leads to sustained oscillations of the vocal cords.
This kind of pressure drop is often referred to as \textit{Bernoulli pressure}, which will be discussed in more depth later.
It is commonly believed that both theories are involved in the production of voiced sounds,
and the ideas are discussed by Titze in ``Comments on the myoelastic-aerodynamic theory of phonation'' \cite[1980]{titze_1980}.
Titze is an extremely prominent author in the field of voice and hearing in mathematical modelling.

The research paper ``Theory and measurement of snores'' \cite[1993]{gavriely_jensen_1993} involves a model which is not designed to approximate phonation.
Rather, its purpose is to provide mechanical insights to the factors present in obstructive sleep apnea.
However, the model itself can be applied to the investigation of phonation,
being a single mass in the wall of a channel subject to stiffness and Bernoulli pressure.
One of the authors, Oliver Jensen, is the supervisor for this project.

Mathematical models for the vocal folds are often heavily simplified, and hence some features of natural phonation are lost.
In ``Synthesis of breathy, normal, and pressed phonation using a two-mass model with a triangular glottis'' \cite[2011]{birkholz_2011},
the authors construct a model for phonation which considers a glottis of a particular shape, which is able to close gradually.
The model is able to synthesise more modes of voiced sounds than traditional models,
particularly being able to produce the range of sounds from ``breathy'' speech (soft-spoken, close to whispering) to ``pressed'' (tense, thin). 

\section{Principles of fluid mechanics}

Euleriean fluid mechanics in a Cartesian coordinate system defines a flow $\mathbf{u}=(u_x,u_y,u_z)$ in a $2$D or $3$D domain,
which is a function of position $\mathbf{r} = (x,y,z)$ and time $t$.
We will describe the $3$D case.
As well as a flow $\mathbf{u}$,
we have fluid density $\rho$ and pressure $p$,
which are often functions of position and time.
In the cases we will study,
we assume the flow is imcompressible,
meaning $\rho$ is uniform across the domain of the flow.
This may appear like a restrictive assumption,
but we still obtain rich and interesting mathematical results from the analysis of an incompressible flow.

The Eulerian framework fundamentally describes the flow field, not the fluid it contains.
A property $\bullet$ of a fluid object must be found using the material derivative
\begin{equation}
    \frac{D\bullet}{Dt} = \frac{\partial \bullet}{\partial t} + (\mathbf{u}\cdot \nabla)\bullet
\end{equation}
where $\nabla$ is the vector differential operator.

A steady flow is a flow $u$ which is independent of time.
The streamlines in a steady flow are fixed, and are identical to the paths taken by a supposed particle placed into a flow at any time.
Bernoulli's equation for a steady flow is as follows:
\begin{equation}
    \frac{1}{2}|\mathbf{u}|^2 + \Omega + \int\frac{\mathrm{d}p}{\rho} = \mathrm{constant~along~a~streamline}.
\end{equation}
The term $\Omega$ is the potential for the body forces on the fluid, such as gravity.
If the potential is either zero or constant,
then $\Omega$ cancels when applying Bernoulli's equation since the body forces $\mathbf{F} = \nabla \Omega$ are zero. %plurals
For our research, we will assume there are no body forces and thus neglect $\Omega$.
Bernoulli's equation is extremely useful, since it gives us a relationship between fluid velocity and pressure in a steady flow.

Fluid mass flux $Q$ at a given cross-section is the rate at which fluid mass passes through the region,
given in terms of an integral:
\begin{equation*}
    Q = -\iint_A \rho \mathbf{u} \cdot \mathbf{n} \mathrm{d}S,
\end{equation*}
where $\mathbf{n}$ is the outer unit normal to a cross-section.
By convention, the outer unit normal points away from the volume,
hence in the opposite direction of the flow, thus we require the negative sign.
This leads to the principle of conservation of mass, where the net flux entering a volume $V$ is the same as the rate of change of fluid mass within that volume.
The net flux is the flux out minus the flow in.
This can be formally expressed as follows
\begin{equation*}
    \iiint_V \frac{\mathrm{d}\rho}{\mathrm{d}t} \mathrm{d}V = - \iint_A \rho \mathbf{u} \cdot \mathbf{n} \mathrm{d}S 
\end{equation*} 

In this research project, the models we will consider involve simplified geometry and more straightforward assumptions of mechanics in comparison to some of the literature we have acknowledged.
Fundamentally, we want to consider a fluid flow through a rigid channel, and suppose that a region of wall has freedom of motion in one dimension normal to the flow.
It is appropriate to introduce the definition of a \textit{plug flow}, being a fluid flow through a channel in which the flow is uniform over cross-sections through the channel.
The plug flow model is a general form for the simple geometry channel flow models we will consider throughout this project.
We will only discuss a case for a two-dimensional flow, since the intricacies are not vital to our analysis.
Assume an incompressible steady flow \(\mathbf{u}(x,y)\) travels through a channel which extends indefinitely in the $x$ direction.
Suppose the channel has a fixed floor at $y=0$ but a variable height $y=h(x)$.
The changes in $h$ affect the flow, since we must maintain conservation of mass.
We can express $Q$ rigorously as an integral
\begin{equation*}
    Q(x) = \int_{0}^{h(x)}u(x,y) \mathrm{d}y
\end{equation*}
and $Q$ must be uniform for any $x$.
We can attempt to find the partial derivative of $Q$ in $x$, which must be zero:
\begin{align*}
    \frac{\partial Q}{\partial x} &= \frac{\partial}{\partial x} \int_{0}^{h(x)}\mathbf{u}(x,y)\mathrm{d}y \\
    &= \int_{0}^{h(x)}\frac{\partial \mathbf{u}}{\partial x}\mathrm{d}y + \left.\frac{\partial h}{\partial x}\mathbf{u}(x,y)\right|_h \tag{by Product Rule,} \\
    &= \left[ \mathbf{u} \right]_0^{h(x)} + \left.
        \left( \mathbf{u}\frac{\partial h}{\partial x} \right)
    \right|_{h_x}.
\end{align*}
We can also find the material derivative of $h$ for a fluid particle:
\begin{equation*}
    \frac{Dh}{Dt} = \frac{\partial h}{\partial t} + U \frac{\partial h}{\partial x}
\end{equation*}
where the remaining terms in the material derivative vanish since all other velocities are negligible.
Incompressibility gives us \(\nabla \cdot \mathbf{u} = \mathbf{0}\), which expands to
\begin{equation*}
    \frac{\partial u_x}{\partial x} + \frac{\partial u_y}{\partial y} = 0.
\end{equation*}

%potentially remove most of this

\section{Principles of non-linear dynamics and dynamical systems}

The modelling of phonation uses techniques from analysing systems of differential equations.
The study of dynamical systems, as far as we are concerned,
involves studying systems of differential equations that describe time-dependence in a model,
and provides insight into characteristics of a model such as equilibrium solutions and classes of behaviours.
In Chapter 2, we study a single mass model to approximate a vocal cord,
and in Chapter 3 we generalise into a two mass model and obtain a fourth order system of differential equations.
In all cases the equations of motion are autonomous and non-linear.
Consider the example on two variables \((x,~y)\), being
\begin{equation}
    \begin{aligned}
        \frac{\mathrm{d}x}{\mathrm{d}t} &= f(x,y) \\
        \frac{\mathrm{d}y}{\mathrm{d}t} &= g(x,y).
    \end{aligned}
\end{equation}
In this example,
the equations are autonomous,
which will be the case for all problems considered in this report.
The \textit{phase portrait} is the representation in which the coordinate axes are the variables $x,y$,
which can be generalised to higher dimensional cases,
but cannot be fully visualised for dimensions higher than three.
We use phase-portrait representation as an alternative visualisation of the behaviour of a system.
If a variable $x$ is governed by an ODE,
we plot its behaviour over time where the different axes represent velocity $dx/dt$ against position $x$,
rather than just visualising position against time.
The phase-portrait is a useful method of representation for problems in dynamical systems,
since an ODE or system of ODEs can be characterised by the locations and properties of its stationary equilibria.
This alternative approach allows us to construct visualisations that show some characteristics of a system more clearly, such as the radii of different closed orbits,
and the positions of equilibrium solutions.
The path we visualise in the phase portrait is a curve of the behaviour of the system parameterised by time $t$. 
Representation of solutions in the phase portrait will be vital to our analysis of our models.

If the equations \(f\) and \(g\) are non-linear,
we can't do much to solve this equation.
Aside from producing numerical computations, we could look intuitively for equilibrium solutions \((x,y) = (x_0,y_0)\),
which are fixed point solutions for which all time derivatives of \(x\) and \(y\) are zero.
Hence equilibrium solutions are equivalently solutions to the homogeneous couple \(f(x,y) = 0,~g(x,y) = 0\).
Close to an equilibrium solution,
we would expect the functions' behaviour to be well approximated by their first order Taylor Series approximations, being
\begin{equation}
    \begin{aligned}
        f(x,y) &\approx f(x_0,y_0) + (x-x_0)\frac{\partial f}{\partial x}(x_0,y_0) + (y-y_0)\frac{\partial f}{\partial y}(x_0,y_0) \\
        g(x,y) &\approx g(x_0,y_0) + (x-x_0)\frac{\partial g}{\partial x}(x_0,y_0) + (y-y_0)\frac{\partial g}{\partial y}(x_0,y_0).
    \end{aligned}
\end{equation}
The constant terms disappear since \(f, g\) are zero at an equilibrium by definition.
As $(x,y) \rightarrow (x_0,y_0)$, the terms of order $n > 1$ approach zero,
and the most significant terms are those involving the first partial derivatives.
We obtain the approximation for the system local to an equilibrium solution,
\begin{equation}
    \begin{aligned}
        \frac{\mathrm{d}x}{\mathrm{d}t} &= \left(\frac{\partial f}{\partial x}(x_0,y_0)\right)(x-x_0) + \left(\frac{\partial f}{\partial y}(x_0,y_0)\right)(y-y_0) \\
        \frac{\mathrm{d}y}{\mathrm{d}t} &= \left(\frac{\partial g}{\partial x}(x_0,y_0)\right)(x-x_0) + \left(\frac{\partial g}{\partial y}(x_0,y_0)\right)(y-y_0),
    \end{aligned}
\end{equation}
which is a system of linear equations.
For convenience, we make the substitutions \(u = x-x_0,~v = y-y_0\) which are linear substitutions proportional to the variables of interest \((x,y)\).
Hence the above approximation is equivalent to the following matrix equation:
\begin{equation}
    \frac{\mathrm{d}}{\mathrm{d}t} \begin{pmatrix}
        u \\
        v
    \end{pmatrix} = \begin{bmatrix}
        \frac{\partial f}{\partial x}(x_0,y_0) & \frac{\partial f}{\partial y}(x_0,y_0) \\
        \frac{\partial g}{\partial x}(x_0,y_0) & \frac{\partial g}{\partial y}(x_0,y_0)
    \end{bmatrix} \begin{pmatrix}
        u \\
        v
    \end{pmatrix}.
\end{equation}
Letting \(\mathbf{u} = (u,v)^\mathrm{T}\) and writing \(J\) as the matrix of derivatives,
we can write the matrix equation compactly as \( \dot{\mathbf{u}} = \mathbf{Ju}\), where the dot denotes the time derivative.
The matrix \(\mathbf{J}\) is the \textit{Jacobian matrix}.
The Hartman-Grobman Theorem in dynamical systems tells us a very important result,
being that the behaviour of the system near the stationary point can be determined by computing the Jacobian matrix \(\mathbf{J}\) at the equilibrium solution,
and computing the eigenvalues of the matrix.
The Jacobian is often a sparse matrix, and its eigenvalues are often complex.
For every equilibrium solution, we must compute the Jacobian and its eigenvalues in order to determine the behaviour of the system at all equilibrium points.
For clarity, the eigenvalues of the Jacobian are the constant terms \(\lambda\) that solve the equation \( \mathbf{J} -  \lambda \mathbf{I} = \mathbf{0}\),
where \(\mathbf{I}\) is the appropriate size identity matrix.
This also concerns the problem of finding equilibrium solutions in the first place, which is usually non-trivial.
Importantly, the vector $\mathbf{u}$ is defined on the independent variables $u,v$ which map one-to-one to the vectors which define the phase portrait.
Hence, eigenvectors of the matrix $\mathbf{J}$ are vectors in the phase portrait from equilibrium solutions,
hence they affect behaviours of the system near these stationary points.

The most important result on eigenvalues of equilibria is that an equilibrium solution is unstable if any of its eigenvalues have positive real part.
Recall that the linearisation of the system is of the form
\begin{equation}
	\dot{\mathbf{u}} = \mathbf{Ju}.
\end{equation}
If $\mathbf{J}$ has an eigenvalue $\lambda$ for an eigenvector $\mathbf{u}_0$,
then we can model the trajectory from this eigenvector by the ODE and initial condition
\begin{equation*}
	\begin{aligned}
		\dot{\mathbf{u}} &= \lambda \mathbf{u} \\
		\mathbf{u}(t=0) &= \mathbf{u}_0.
	\end{aligned}
\end{equation*}
This is a first order vector ODE and has a general solution
\begin{equation}
	\mathbf{u} = \mathbf{u}_0 e^{\lambda t}
\end{equation}
and hence the value of $\lambda$ determines the trajectory from the initial point $\mathbf{u}_0$.
If $\lambda$ has positive real part,
then the trajectory diverges exponentially from the equilibrium,
and hence the equilibrium is unstable.

For a more rigorous understanding of the linearisation of a system of equations and an understanding of the Hartman-Grobman Theorem,
see \cite{perko_textbook_1996},
particularly sections 1.1 on simple examples of linear systems,
1.5 on two-dimensional linear systems,
1.9 on the theory of stability,
and 2.6 on linearisation.

\chapter{Writing go here}

\dots

\chapter{Review}

\section{Energy}

Both the single mass and the two mass model are systems which perfectly preserve energy.
This is not consistent in the computations,
but we have shown for both models that there exists a constant term,
depending on the variables of position and velocity,
which is unchanging in time.
From a realistic perspective,
systems tend to dissipate energy and eventually come to rest,
rather than behaving like perfect conservative machines that operate indefinitely.

Both models we have studied lack any form of damping,
so all potential energies are perfectly transferred to kinetic energies and vice versa.
If we were to introduce a damping parameter,
we would observe closed oscillations about a stationary point to always converge to that equilibrium,
rather than oscillating indefinitely.

A mass $m$ attached to a Hooke spring obeys the ODE
\begin{equation*}
	m\frac{\mathrm{d}^2 x}{\mathrm{d}t^2} + k(x-x_0) = 0
\end{equation*}
where $x$ is the length of the spring, $x_0$ is the length at rest,
$k$ is the stiffness,
and $t$ is time.
This is the equation for a simple harmonic oscillator,
with equilibrium solution $x=x_0$.
The general solution is the expression
\begin{equation*}
	x = x_0 + A\exp \left(
		\sqrt{-\frac{k}{m}}t
	\right) + B\exp \left(
		-\sqrt{-\frac{k}{m}}t
	\right),
\end{equation*}
where $A$ and $B$ are free constants.
With the inclusion of a damping parameter,
we obtain a different ODE
\begin{equation*}
	m\frac{\mathrm{d}^2 x}{\mathrm{d}t^2} + c\frac{\mathrm{d}x}{\mathrm{d}t} + k(x-x_0) = 0
\end{equation*}
where the constant $c$ describes the damping strength.
We retain the equilibrium solution $x=x_0$, but the general solution changes to
\begin{equation*}
	x = x_0 + A \exp \left(
		\frac{-c+\sqrt{c^2-4mk}}{2m}t
	\right) + B \exp \left(
		\frac{-c-\sqrt{c^2-4mk}}{2m}t
	\right).
\end{equation*}
We require $m,~k,~c$ to be positive constants.
The harmonic oscillator is periodic,
whereas the damped harmonic oscillator eventually decays to rest at the equilibrium.

Our models describe systems where energy is perfectly transferred between useful forms,
namely potential and kinetic energies.
In other words, the systems we study have $100\%$ efficiency.
In reality, there are virtually no dynamical systems with this level of efficiency,
rather that some energy is always wasted.
If would be interesting to develop our models by introducing forms of energy dissipation,
such that our models would not describe systems which are $100\%$ efficient.
We could then consider how the forcing pressure could adapt to continue phonation to take place when the motion of the vocal cords begins to decay towards equilibrium.

% talk about voice and natural systems that lose energy

\section{Elasticity}

From studies into the material properties of the vocal cords,
they have been observed to deform under stress in a very nonlinear elastic way \cite{alipour_2012}.
However, in the models we have studied,
the stiffness of the vocal cords are modelled by a Hooke spring with linear elasticity.
It is important to note that in the two mass model,
we simplified these assumptions further,
by approximating the stiffness coupling by a linear force,
rather than as the vertical component of a diagonal spring.

The purpose of our mathematical models are to describe the potentially intricate motion of the vocal cords.
In order to do this,
we used Hooke springs to model the stiffnesses.
While not ideal, these are fundamental to the quasiperiodic behaviours we observed in the two mass model.
If we considered non-linear stiffnesses,
these might provide fundamentally different oscillations to our observations.
However, our analysis does not extend to the impacts of different spring behaviours on the results.
Rather, we made the approximation of linear Hooke springs and analysed their impact in depth.

It also serves to mention that the linear Hooke spring is a first-order approximation to the stiffness of a real spring.
As such, it is reasonably accurate for small displacements,
which occur in the proximity of stable equilibria,
being a large portion of our analysis.

\section{Stablility of fluid flow}

In both models,
we applied Bernoulli's equation for a steady flow in order to deduce a relationship between the velocity and the pressure of the fluid.
It is possible for the spoken sounds in phonation to be produced by a steady flow,
but phonation can occur more generally for unsteady, turbulent flows,
and this is not a case which is accommodated in our modelling.

In the two mass model,
we forced the assumption of a quasisteady flow,
however this reduced the cases we accommodated in the modelling.
Ideally we would construct a model which provides accurate simulation of vocal folds moving, subject to a turbulent flow through the glottis.
However, the quasisteady flow still provided rich results which led to a wide variety of behaviours which we were able to analyse.
Were we to develop the model and introduce the non-steady flow,
it would be very important to replicate results from the quasisteady flow assumption.
This is because if a model cannot retain a result when a component changes,
then there may be something fundamental to that original component,
which we have now lost.
As such,
it was important to focus on the quasisteady flow assumption,
even if this does not capture the range of flow types that may be involved in phonation.

\section{Physical structure of the model}

The formulation of the two mass model involved extremely simple geometry,
being the flow through a rectangular channel.
A more varied and irregular structure,
similar to the interior shape of the larnyx and glottis,
could potentially influence the dynamics of the fluid flow.
A varying inner channel width could have implications in Bernoulli flow,
and could contribute to more complex flow patterns in phonation.

The masses which model the vocal cord are extremely simple objects,
being stiffness-coupled planes,
and are restricted to one dimensional motion in the direction of their ourward normal.
In reality, the vocal folds have a much more intricate shape than the cuboid blocks we modelled them as.
We could develop the two mass model by constructing the vocal fold with a much more gemoetrically complex shape,
and allowing more degrees of freedom in their motion.
This would lead to more complex behaviour in the motion of the masses,
but it would be difficult to recover results from the simpler model which we have analysed.
This is because the ODE we have analysed in the two mass model is characteristic of the one dimensional motion.
If we were to remove this restriction,
we would obtain a system of PDEs describing motion in more than one degree of freedom,
and it would be difficult to replicate the one dimensional results from our two mass model.


\chapter{Conclusion}

In this project,
we have explored two models for phonation and evaulated their accuracy and utility as mathematical models.
We started by introducing the fundamentals of fluid mechanics and mathematical modelling,
which build the foundation of our modelling process.




\bibliographystyle{unsrt} % agsm is harvard
\bibliography{sources}

\appendix

%% bibliography styles
% agsm is harvard
% unsrt is standard numerical
% ieeetr is whatever jasmine uses

\chapter{MATLAB Scripts}

\section{Single mass model functions}

In MATLAB, we define the ODE function \texttt{OscillatorODE(t,x,q,u)} as follows:

%nicer written verbatim environment ode function here

\begin{verbatim}
	function dbdt = OscillatorODE(t, x, mu, q)
	
	b=x(1);
	bdot=x(2);
	
	dbdt=zeros(size(x));
	dbdt(1) = bdot;
	dbdt(2) = 1 - q - b - (mu*power(q,2))/(2*power(b,2));
	
	end
\end{verbatim}

And it is solved with \texttt{ode45} as such:

\begin{verbatim}
	q=0.1; u=1; % define fixed variables
	[t, x] = ode45(@(t,x) OscillatorODE(t,x,q,u));
	
	% distance-time plot
	figure
	plot(t,x(1,:))
	
	% phase-plane plot
	figure
	plot(x(1,:),x(2,:))
\end{verbatim}

\end{document}