\documentclass{book}

\usepackage{amssymb}
\usepackage{amsfonts}
\usepackage{amsmath}
\usepackage{dsfont}
\usepackage{bm}
\usepackage[a4paper, total={6in,8in}]{geometry}
\usepackage{graphicx}
\usepackage{float}
\usepackage{natbib}
\usepackage{hyperref}

\graphicspath{{figures/}}

\title{MATH30000 Project - Phonation}
\author{Will Woolfenden}

\begin{document}

\maketitle

\chapter{Introduction}

\dots

\section{Principles of non-linear dynamics and dynamical systems}

The modelling of phonation uses techniques from analysing systems of differential equations.
The study of dynamical systems, as far as we are concerned,
involves studying systems of differential equations that describe time-dependence in a model,
and provides insight into characteristics of a model such as equilibrium solutions and classes of behaviours.
In Chapter 2, we study a single mass model to approximate a vocal cord,
and in Chapter 3 we generalise into a two mass model and obtain a fourth order system of differential equations.
In all cases the equations of motion are autonomous and non-linear.







\end{document}