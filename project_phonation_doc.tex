\documentclass{article}

\usepackage{amssymb}
\usepackage{amsfonts}
\usepackage{amsmath}
\usepackage{dsfont}
\usepackage[a4paper, total={6in,8in}]{geometry}
\usepackage{hyperref}
\usepackage{graphicx}
\usepackage{biblatex}

\addbibresource{source  s.bib}

\title{MATH30000 Project - Phonation}
\author{Will Woolfenden}

\begin{document}

\maketitle

\section{Introduction}

\subsection{Definitions}

Phonation is the human process in which respiratory exhalation and vocal cord oscillations work together to produce sounds that we identify as spoken words.
Air is projected from the lungs in a fluid motion through the airways.
When passing through the larynx, the forced oscillations of the contained vocal cords interact with the air particles and induce the sound of speech.

The goal of this project is to explore models of the phonation process.



\section{Model}

\subsection{Derivation}

The model in Theory and Measurement of Snores \cite{gavriely_jensen_1993} models inspiratory respiration.
The fluid flow $\dot{V}$ enters through the upper airway, which has flow resistance $R_{ua}$,
before passing through a constricted volume with a wall at a variable position.
The flow then leads out of the constricting region to the lungs.

Principles of fluid motion lead to families of behaviour exhibited by the channel wall,
where we denote the distance between the opposing walls as $b$.
In a steady flow, along a streamline, Bernoulli's equation is defined as 

\begin{equation}
    \frac{1}{2}\mathbf{u}\cdot\mathbf{u} + \Omega + \int\frac{dp}{\rho} = C
    \label{eqn:bernoulli}
\end{equation}

where $\frac{1}{2}\mathbf{u}\cdot\mathbf{u}$ is the fluid kinetic energy, 
$\Omega$ is the external potential energy,
and $\int\frac{dp}{\rho}$ is the internal potential energy\footnote{All energy is per unit mass}.

Clearly if the wall constricts, reducing the volume of the fluid flow,
then the pressure $p$ of the flow $u$ increases,
and by \ref{eqn:bernoulli} the kinetic energy must increase.

%the bernoulli equation implies that a change in velocity leads to an opposite-oriented change in pressure?
%the model discusses a change in v caused by b, and a change in b caused by v separate.

%discuss more in depth on the derivation of the forces

We end up with the governing equation for the channel wall positive displacement from collapse at $0$,

\begin{equation}
    \frac{d^2b}{dt^2} = 1 - q - b - \frac{\mu q^2}{2b^2}
    \label{eqn:master}
\end{equation}

where $\mu, q$ are parameters linked to the forces acting on the channel wall due to flow velocity and resistance.

\subsection{Matrix Method}

We start by analysing parts of the function separately.
Our goal is to understand how oscillations in $b$ occur.
First consider defining a function on $b$ for the RHS of Equation \ref{eqn:master},
namely

\begin{equation}
    f(b) = 1 - q - b - \frac{\mu q^2}{2b^2}.
\end{equation}

Clearly $f$ represents acceleration of the movable wall as a function of displacement $b$.
We are only ever concerned with positive $b$ due to the assumptions of the model.
The function $f$ tends towards negative infinity both as $b\rightarrow 0$ and as $b\rightarrow\infty$,
but $f'$ is monotone decreasing, so there may be an interval $I_0=(p,q)$ such that if $b\in I_0$
then $f$ is positive i.e. $p,q$ are the zeroes of $f$.
It is not certain that $f$ will have a positive region,
and this is decided by the parameters $\mu$ and $q$.
Particularly, we require that the maximum of $f$ ($f'(b_c) = 0$) is greater than zero:

%definitely need to include a plot of f(b) in the above passage

\begin{align}
    f(b) &= 1 - q - b - \frac{\mu q^2}{2b^2} \\
    \Rightarrow f'(b) &= -1 + \frac{\mu q^2}{b^3}
\end{align}

From which we can deduce that the maximum is located at the point where $b^3 = \mu q^2$.
If we plug this back into $f$, we can deduce the condition.

\begin{align*}
    f((\mu q^2)^{\frac{1}{3}}) &= 1 - q - (\mu q^2)^{\frac{1}{3}} - \frac{\mu q^2}{2(\mu q^2)^{\frac{2}{3}}} \\
    &= 1 - q - \frac{3}{2}(\mu q^2)^{\frac{1}{3}}
\end{align*}

We obtain the result

\begin{equation}
    1 - q - \frac{3}{2}(\mu q^2)^{\frac{1}{3}} > 0,
    \label{eqn:osc_condition}
\end{equation}

being the requirement for there to exist an interval $I_0$
such that if $b\in I_0$ then $f$ is positive.
In other words, if this condition is met then oscillations may occur.
Further on, we will be regularly making the assumption that this condition is satisfied,
and hence that oscillations can occur,
in order to develop our analysis of the model.

If the acceleration of the channel wall is positive, then the wall will accelerate outwards,
however the acceleration will have to oscillate from positive to negative in order for there to be oscillations in the position itself.

\subsection{Numerical Computations}

To formulate this model into a computation, we reduce it into a first-order system of differential equations

\begin{align}
    \frac{db}{dt} &= \dot{b} \\
    \frac{d\dot{b}}{dt} &= 1 - q - b - \frac{\mu q^2}{2b^2}
\end{align}

and this can be inspected to find new features of the behaviour of the model.
First of all, $\frac{d\dot{b}}{dt} = f(b)$.
We can propose an approximation of $f$, of the form

\begin{equation*}
    f(b) = f(b_0) + (b-b_0)f'(b_0) + \frac{(b-b_0)f''(b_0)}{2!} + \mathellipsis = \sum_{k=0}^\infty\frac{(b-b_0)^{k}f^{(k)}(b_0)}{i!}
\end{equation*}

namely, the Taylor series expansion of $f$ around a point $b_0$.
Pick $b_0$ to be a zero of the function $f$\footnote{consequently this model provides analysis which is not valid elsewhere on the model.}
then we find that the $f(b_0)$ term vanishes.
Furthermore, as $b$ tends to $b_0$, the higher order terms also vanish,
and with reasonable tolerance we are left with the approximation

\begin{equation}
    f(b) = (b-b_0)f'(b_0).
\end{equation}

leading us to an approximation for the system of differential equations

\begin{align}
    \frac{db}{dt} &= \dot{b} \\
    \frac{d\dot{b}}{dt} &= (b-b_0)f'(b_0).
\end{align}

The motivation for this approximation can now be expressed,
being that with the substitution

\begin{align}
    U &= b - b_0 \\
    V &= \dot{b}
\end{align}

we can write the system of differential equations in matrix analysis,
allowing us to apply methods on solutions and analysis of linear systems.

\begin{equation}
    \frac{d}{dt}\begin{pmatrix}
        U \\
        V
    \end{pmatrix} = \begin{bmatrix}
        0 & 1 \\
        f(b_0) & 0
    \end{bmatrix} \begin{pmatrix}
        U \\
        V
    \end{pmatrix}
    \label{eqn:matrix}
\end{equation}

We now have a matrix approximation for the differential operator $\frac{d}{dt}$.
The matrix and vectors are defined in the vector space spanned by the vectors $U,V$,
and so the eigenvalues of this matrix are directly related to vectors in the $b,\dot{b}$ vector space,
which itself is referred to as the \textit{phase portrait}.
And if the matrix has eigenvectors, we can inspect those eigenvectors (which are local to the point $b=b_0$),
allowing us to evaluate the behaviour of the model, in the phase-plane,
when local to $b=b_0$.

Note that this analysis is governed by being local to the zeroes of $f(b)$.
It only provides insight on the behaviour of the model when considering these points.
When oscillations occur, it is a consequence of the condition of Equation \ref{eqn:osc_condition},
which is the condition of $f(b)$ having two positive zeroes.
Thus, the matrix equation analysis is only valid at two points,
being these solutions of $f(b)$ at the bounds of oscillation.


\subsection{Eigenvectors in the Phase-Plane}

The phase-plane environment is used to display position and velocity as both parametrically defined in time.
We visualise $b$ and $\dot{b}$ as orthogonal vectors, and plot curves to visualise position and velocity parametrically.

%phaseplane figure%

The matrix $\begin{bmatrix}
	0 & 1 \\
	f'(b_0) & 0
\end{bmatrix}$ has eigenvectors $\mathbf{v}$ in the vector space $V = \operatorname{span}\{(b-b_0),\dot{b}\}$.
The path in the phase-plane from $b=b_0$ will follow the trajectory of this eigenvector at the initial time.

%saddle and node graph%

The zeroes of $f$, where the matrix method is valid\footnote{to a similar extent as all other approximations},
correspond to the saddle and node points represented in the figure above. %amend reference to the figure.
Clearly, oscillations can occur in an interval of initial $b$ that starts at the first positive zero of $f(b)$,
but extends by some region past the second zero.
If helps to perform some more analysis on the governing equation from a mechanics perspective.

\subsection{Further Mechanics}

Recall Equation \ref{eqn:master}, of the form $\frac{d^2b}{dt^2} = 1 - q - b - \frac{\mu q^2}{2 b^2} = f(b)$.
Assume the right hand side $f(b)$ is the derivative (with respect to $b$, remaining aware that $b$ depends on $t$) of some function $P(b)$,
so $\frac{d}{dt}\left(P(b)\right) = f(b)\frac{db}{dt}.$ %some of this might be abusive. consider in terms of d/db
Multiply both sides of the equation by $\frac{db}{dt}$ to obtain

\begin{equation*}
    \frac{d^2b}{dt^2} \frac{db}{dt} = f(b) \frac{db}{dt}.
\end{equation*}

The definition of $P(b)$ means we can integrate the RHS trivially,
and we can also integrate the left hand side by a reverse of the product rule for differentiation.
Hence, we obtain

\begin{equation}
    \frac{1}{2}\left(\frac{db}{dt}\right)^2 = P(b) + C
    \label{eqn:integral_kinetic_energy}
\end{equation}

for an arbitrary constant $C$. This represents a family of solutions.
Notice that we can interpret the left hand side as $\frac{1}{2}v^2$ for a velocity $v$;
this expression is proportional to the kinetic energy of the wall.

We already have a direct expression for $f$,
so we can use our knowledge of the $f(b)$ vs $b$ graph and integrate $f(b)$ with respect to $b$
in order to obtain an explicit $P(b)$.

Extremely important to note is that we have an equation in terms of $\frac{db}{dt}$ and $b$, which are the vectors defining the phase portrait.
Hence the curves that appear in the phase portrait represent all the curves that appear for different $+C$.

%figure with the $P(b)$ curves matched to phase portrait curves

This provides analytical means for inspecting the range of oscillations.

We must be careful with considering how we are able to do this, and how it provides valid analysis.
The Equation \ref{eqn:master} is autonomous,
so the same initial conditions will always exhibit the same behaviour regardless if we start at $t=1$ or $t=100$.
When we graph $f$ and $P$ and inspect them analytically, we are considering their behaviour as affected by the parameters $mu$ and $q$.
Autonomy allows us to graph $f$ and $P$ over $b$ and analyse.

\subsection{Oscillations and the Parameter Space}

Recall Equation \ref{eqn:osc_condition}. In the derivation of the governing equation of the model, $\mu$ and $q$ are strictly positive\footnote{They represent forces related to the viscous resistance and the changes in pressure. The properties of these parameters will be discussed later}.
To aid our understanding, consider the \textit{parameter space,} i,e, the vector space spanned by vectors $\mu, q$.
Any specific form of Equation \ref{eqn:master} belongs to a point in the parameter space,
and we can deduce that its position in the space influences the range of initial $b$ for which oscillations will occur. 

% graphs of KE, d2b, for different mu, q

The function $P(b)$ is defined as the definite integral of $f(b)$ with respect to $b$.

\begin{equation}
    \begin{aligned}
        P(b) &= \int f(b) db \\
        &= \int 1 - q - b - \frac{\mu q^2}{2 b^2} db \\
        &= (1-q)b - \frac{1}{2}b^2 + \frac{\mu q^2}{2b}.
    \end{aligned}
    \label{eqn:ke_integral}
\end{equation}

A curve $P(b)$ is a continuous function of $b$ such that $\frac{1}{2}\left(\frac{db}{dt}\right)^2 = P(b) +C$ for all $C \in \mathds{R}$
represents all curves in the phase portrait. Therefore, the shape of $P(b)$ and the location of its critical points (turning points and zeroes)
represent the range in which oscillations will occur.

Given $\mu, q$ satisfactory for oscillations to occur,
it is evident that there exists a $C$ such that $P(b)+C$ will have exactly two zeroes,
and this interval is exactly the region in which we observe the largest oscillation,
being which the first zero is a point observed in our analysis and derivation of Equation \ref{eqn:matrix}.

Given $P(b)$ as defined in Equation \ref{eqn:ke_integral},
our query is to find the case where the following equation is satisfied:

\begin{equation}
    (1-q)b - \frac{1}{2}b^2 + \frac{\mu q^2}{2b} + C = 0,
\end{equation}

such that there exist solutions for exactly $2$ values of $b$.

Since oscillations are guaranteed to occur (because we assumed so),
we can guarantee that \ref{eqn:osc_condition} is satisfied.

% derivation of the conditions on +C which guarantee the closed orbit. 

% rigorous understanding of the range of oscillations


% behaviour of oscillations local to the "matrix points", tie in the analysis to the matrix methods.

% comparison of phaseplane and distance-time graphs



\end{document}