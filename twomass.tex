\documentclass{article}

\usepackage{amssymb}
\usepackage{amsfonts}
\usepackage{amsmath}
\usepackage{dsfont}
\usepackage{bm}
\usepackage[a4paper, total={6in,8in}]{geometry}
\usepackage{graphicx}
\usepackage{float}
\usepackage{natbib}
\usepackage{hyperref}

\graphicspath{{figures/}}

\begin{document}

\title{A Two Mass Model}
\author{Will Woolfenden}

\section{Development of a two-mass model}

We will begin to construct a model for phonation involving two stiffness-coupled masses.
Due to symmetry, we will only consider one side of the channel, 
We consider a steady flow $\mathbf{u}$ passing through a channel,
in which two masses cause a constriction to the fluid passage.
Each mass $m_i$ is indivually supported by a spring with Hooke constant $k_i$,
and a coupling spring with constant $k_s$ connects the two masses.

We assume the masses to move one-dimensionally, perpendicular to the principal direction of the flow,
such that they may extend indefinitely.
We make the assumption that the fluid travels similar to a plug flow approximation,
in which the fluid velocity is constant across any cross-sectional area, % citation here - there's a text "fundamentals of fluid mechanics 9780471675822"
which ignores any potential stagnation or interference.
We also neglect external forces on the fluid, such as gravity.

Air is expelled from the lung and into the upper airway, before being released into the atmosphere.
We fix a pressure $p_0$ and velocity $U_0$ local to the lung.

The formal expression for conservation of mass can be expressed as

\begin{equation}
    \iiint_V \frac{\partial\rho}{\partial t}dV = \iint_S -\rho \bm{u\cdot n}dS,
    \label{eqn:cons_mass_formal}
\end{equation}

which can be more formally understood as the rate of change of mass over volume $V$ begin equal to the rate at which fluid is transferred over the volume's surface $S$. 
The negative term comes from the convention of using the outer unit normal of a closed surface.
We will refer to the \textit{flux} at $A$, denoted $Q$, being the rate at which mass of fluid is transferred over an area $A$, or formally:

\begin{equation}
    Q(x) = -\iint_A \rho \bm{u \cdot n}dS
\end{equation}

where $x$ is the position in the dominant direction of the flow, or equivalently $x$ is in the $\mathbf{n}$ direction.
The outer unit normal $\mathbf{n}$ is the vector pointing outwards normal to any cross-sectional area. 
The integral term is negative because the outer unit normal points in the direction opposite to the flow.

We can find the flux $Q_0$, being the flux local to the lung, from the terms $U_0,~p_0$ which are defined in our initial conditions,

\begin{equation}
    Q(x_0) = -\iint_{A_0} \rho \bm{u}(x_0)\bm{\cdot n}(x_0)dS.
\end{equation}

Since we have a condition $u(x=x_0) = U_0$, we can write this as

\begin{equation}
    Q(x_0) = \iint_{A_0} \rho U_0 dS.
\end{equation}

We have made the assumption of incompressibility in order to write $\rho$ as a constant.
Since $A_0$ is a cross-section of known area $wh$, and $U_0$ is known, we have

\begin{equation}
    Q(x_0) = \rho wh U_0
    \label{eqn:twomass_lung_flux}
\end{equation}

by evaluating the integral of a uniform quantity $U_0$ over a surface $A_0$ as the quantity multiplied by the whole area.
The flux is not uniform, because we have regions of variable volume, and a varying flux local to these regions do not break assumptions of conservation of mass.
For example, if more fluid is entering a region than what is leaving, then the volume will increase.
We have flux $Q_{in}$ on the entry border of the two mass region, and $Q_{out}$ on the exit,
with $Q_{mid}$ on the plane in between the masses.

A volume $V_i$ has dimensions $h_i w d$, where $h_i + b_i = h$ for $i=1,2$.
The height $h_i$ is the only parameter that may vary, since we have granted a plate with position $b_i$ one dimensional motion.
We will take time derivatives:

\begin{equation}
    \begin{aligned}
        V_i &= h_i w d \\
        \dot{V_i} &= \dot{h_i} w d.
    \end{aligned}
\end{equation}

Conservation of mass \ref{eqn:cons_mass_formal} in combination with incompressibility tells us that the rate of change of a volume is equal to the rate at which its enclosed mass increases.
In volume $V_1$, mass enters through flux $Q_in$ and exits through flux $Q_mid$. The case for $V_2$ is analogous.
From this we can force an assumption to relate flux around a volume to its rate of change, which can be expressed as follows:

\begin{equation}
    \begin{aligned}
        \dot{h_1} w d &= Q_\mathrm{in} - Q_\mathrm{mid} \\
        \dot{h_2} w d &= Q_\mathrm{mid} - Q_\mathrm{out}.
    \end{aligned}
    \label{eqn:twomass_flux_motion}
\end{equation}

Since the lower airway region is fixed,
the flux $Q_0$ local to the lung is equal to the flux $Q_{in}$ on the boundary of the lower mass,
and hence it is a known term.

% We want to know the values of Q_mid and Q_out so we can interpolate


The formal definition of incompressibility can be expressed as

\begin{equation}
    \bm{\nabla\cdot u} = 0.
\end{equation}


\end{document}

