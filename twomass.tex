\documentclass{article}

\usepackage{amssymb}
\usepackage{amsfonts}
\usepackage{amsmath}
\usepackage{dsfont}
\usepackage[a4paper, total={6in,8in}]{geometry}
\usepackage{graphicx}
\usepackage{float}
\usepackage{natbib}
\usepackage{hyperref}

\graphicspath{{figures/}}

\begin{document}

\title{A Two Mass Model}
\author{Will Woolfenden}

\section{Development of a two-mass model}

We will begin to construct a model for phonation involving two stiffness-coupled masses.
Due to symmetry, we will only consider one side of the channel, 
We consider a flow $\mathbf{u}$ passing through a channel,
in which two masses cause a constriction to the fluid passage.
Each mass $m_i$ is indivually supported by a spring with Hooke constant $k_i$,
and a coupling spring with constant $k_s$ connects the two masses.

We assume the masses to move one-dimensionally, perpendicular to the principal direction of the flow,
such that they may extend indefinitely.
We make the assumption that the fluid travels similar to a plug flow approximation,
in which the fluid velocity is constant across any cross-sectional area, % citation here - there's a text "fundamentals of fluid mechanics 9780471675822"
which ignores any potential stagnation or interference.
We also neglect external forces on the fluid, such as gravity.

Air is expelled from the lung and into the upper airway, before being released into the atmosphere.
We fix a pressure $p_0$ and velocity $U_0$ local to the lung.

The formal expression for conservation of mass can be expressed as

\begin{equation}
    \iiint_V \frac{\partial\rho}{\partial t}dV = \iint_S -\rho \mathbf{u}\cdotp\mathbf{n}dS,
    \label{eqn:cons_mass_formal}
\end{equation}

which can be more formally understood as the rate of change of mass over volume $V$ begin equal to the rate at which fluid is transferred over the volume's surface $S$. 
The negative term comes from the convention of using the outer unit normal of a closed surface.
We will refer to the \textit{flux} at $A$, denoted $Q$, being the rate at which mass of fluid is transferred over an area $A$, or formally:

\begin{equation}
    Q(x) = \iint_A \rho \mathbf{u}\cdotp\mathbf{n}dS
\end{equation}

Where $x$ is the position in the dominant direction of the flow, or equivalently $x$ is in the $\mathbf{n}$ direction.

We can find the flux $Q_0$, being the flux local to the lung, from the terms $U_0,~p_0$ which we fix in initial conditions.


\end{document}

